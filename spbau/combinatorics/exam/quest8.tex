\chapter{Вопрос №8}

Производящие функции - основные определения, операции, свойстваю Простейшие примеры.

\section{Основные определения}

Пусть дана проивзольная последовательность чисел $$ a_0, a_1, a_2, ... $$
Формальный степенной ряд вида $$ A\left(x\right) = a_0 + a_1x + a_2x^2 + ... = \sum_{k=0}^\infty a_k x^k$$
называется обыкновенной производящей функцией (опф) последовательности $\left\{a_k\right\}$, а формальный степеной ряд вида: $$ E\left(x\right) = a_0 + a_1 x + a_2 \frac{x^2}{2!} + ... = \sum_{k=0}^\infty a_k\frac{x^k}{k!} $$
называется экспоненциальной производящей функцией (эпф) для последовательности.

В данном случае выражение формальный степенной ряд означает, что его не стоит рассматривать как ряд в классическом смысле (с определениями сходимости и тд и тп)

Для последовательности $\left\{k!\right\}$ опф будет выглядеть так: $$ \sum_{k=0}^\infty k! x^k $$

В силу того, что ряд формальный, $x$ не обязательно рассматривать как некоторую числовую переменную, а можно придавать ей проивзольный смысл, в зависимости от требований задачи.

\section{Основные операции}

Суммой двух производящих функций $A\left(x\right) = \sum_{k=0}^\infty a_k x^k$ и $B\left(x\right) = \sum_{k=0}^\infty b_k x^k$ называется производящая функция $C\left(x\right) = \sum_{k=0}^\infty c_k x^k$, для которой $$c_i = a_i + b_i$$ (аналогично и для эпф).

Если рассматривать опф как полиномы и перемножить их, то получится, что произведение опф $A\left(x\right)$ и $B\left(x\right)$ называется опф $C\left(x\right)$, для которой $$ c_i = \sum_{k=0}^i a_kb_{i-k}$$

Аналогичные действия для эпф приводят к другой формуле $$ c_i = \sum_{k=0}^i \binom{i}{k} a_kb_{i-k} $$

Эти операции ассоциативны и коммутативны.

Кроме операции обычного умножения есть еще и операция почленного умножения числовых последовательностей, в случае пф она называется произведением Адомара.

\section{Примеры}

Пусть имеется: $$ \left(\sum_{n=0}^{\infty} \alpha^nx^n\right)\left(1-\alpha x\right) = \sum_{n=0}^{\infty} c_n x^n $$ чему равны коэффициенты $c_n$? Очевидно, что $c_0 = 1$, а все последующие $c_i = 0, i>0$, т. е. $\left\{c_n\right\}$ - нейтральная числовая последовательность относительно свертки последовательностей, а соответствующа п. ф. является нейтральной относительно умножения п. ф. а тогда:
\begin{equation}
	\sum_{n=0}^{\infty} \alpha^n x^n = \frac{1}{1 - \alpha x}
\end{equation}

данный пример показывает, что хоть мы и рассматриваем формальные степенные ряды, некоторые тождества для них совпадают с тождествами для обыкновенных степенных рядов (хотя в случае обыкновенных степенных рядов, это тождество справедливо только в области сходимости ряда).

Для последовательности $\left\{\binom{n}{k}\right\}$ опф хорошо известна:
\begin{equation}
	f\left(x\right) = \left(1+x\right)^n = \sum_{k=0}^{\infty} \binom{n}{k}x^k = \sum_{k=0}^{\infty} \Lowerfact{n}{k} \frac{x^k}{k!}
\end{equation}
т. е. она же является эпф для последовательности $\left\{\Lowerfact{n}{k}\right\}$
